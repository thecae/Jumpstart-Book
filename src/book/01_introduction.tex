\documentclass[../book.tex]{subfiles}
\graphicspath{{\subfix{../images/}}}

\begin{document}
\chapter{Introduction}
Welcome to Suncoast High School!  We are very glad that you chose us and hope that you enjoy your next four years here as an MSE student.  The Math, Science, and Engineering program is a mathematically rigorous program that offers classes in college-level mathematics, the sciences, and physics.  It prepares students to perform at the very best in college and beyond and is proud to guide students to admission at the nation's best universities, including every Ivy League, MIT, Stanford, Vanderbilt, Johns Hopkins, UF, and more.

The goal of this text is to offer a pre-Suncoast curriculum to rising Freshman.  This text will review the important Algebra II skills and cover fundamental skills necessary for physics.  We hope that each student takes advantage of the textbook, even if some of the material is not covered during the week-long session.

In writing this text, we were forced to leave out some steps in solving problems due to space limitations and to maintain the conciseness of the book.  We hope that students show all work when solving problems to minimize the mistakes made during assessments.  In later years of mathematics, some steps may be left out to allow room for solving longer and more complex problems; however, at this level, showing all steps is necessary.

\section{Study Habits}
Suncoast, at many points, can have very stressful moments throughout the four years.  We hope that instilling various study habits will allow you to perform as best as possible to ensure adequate mental and physical health, emotional stability, and maintain self-confidence levels.  Below is an incomplete list of study habits and advice that you should follow during your time at Suncoast: \begin{itemize}
    \item \textit{Find a study group}.  One of the most important things to have at Suncoast is to have a trusted friend group to which you can ask questions, work on homework together, and study for tests.  Remember, multiple minds are greater than one!
    \item \textit{Online help sites don't help}.  If you're considering using Slader, Chegg, CourseHero, Yahoo Answers, or other help sites to find the answers to your homework, you're not helping yourself.  Students often exploit these websites to quickly complete their homework before the due date rather than starting their homework earlier, understanding the material, and working on the assignment without these sites.  Indeed, it's important to ask questions if you don't understand something; however, using these sites to do your entire homework isn't a good idea by any means.
    \item \textit{Minimize Procrastination}.  As former Suncoast students, we understand procrastination! Please learn from our mistakes, don't make a habit of putting things off until the last minute.  Given the far larger work load you will inevitably experience through your four years here, it is highly recommended that you try to minimize procrastination.  If you ever have a large assignment due, never wait until the day before to start; you could end up losing significant amounts of sleep, even pulling an all-nighter, harming your cognition the next day.
    \item \textit{Take advantage of tutoring}.  Suncoast has many, many tutoring opportunities around campus during lunch and after school.  Mu Alpha Theta offers one-on-one math and physics tutoring from Monday through Thursday in various classrooms.  At some point near the start of the year, the school sends a list of teachers from every department that offer tutoring after-school to ask questions and review material.  Find these opportunities and attend them at the first realization of difficulty; don't wait until your grade suffers before you ask for help.
    \item \textit{Stay Organized}.  There is no generalized solution for this but find a system that works for you.  Personally, we have found it best to keep your classes in separate notebooks/folders/binders.  Teachers will all ask for binders, but sometimes this isn't totally necessary; if they require a binder (meaning they're doing binder checks), then get the binder.  If they say binders are optional, ask them about the number of handouts - the less handouts means the lesser need for a binder.
\end{itemize} 


\section{How to Read This Text}
It is your choice how to read this text; it was intended to be read from cover to cover in order.  If you find that you want to explore one topic, feel free.  Unless a chapter title is explicitly related to a previous, it shouldn't require previous material and knowledge.  If you don't understand something, backtrack until you find the information you don't understand.

\begin{remark}
  It is not recommended to read the second half until you've completed the first half.  The second half is entirely based on the first, so it is inevitable that you will find parts that you don't understand.  
\end{remark}
\subsection{Organization}
\noindent You can think of this book as being divided into three different sections: \begin{enumerate}
    \item \textit{Chapters 2 through 9}.  These chapters form the core of the Jumpstart material.  They are the basis of the Algebra II knowledge required for your time at Suncoast.  It covers most the material that Algebra II typically covers and then physics skills to ensure that students are comfortable in both courses.
    \item \textit{Chapters 10 through 13}.  This material is material that may have been covered in some Algebra II courses.  This is not an essential part of the Jumpstart curriculum, as it is re-taught in Pre-Calculus.  For those that want to get a head-start on this material, feel free to look it over.
    \item \textit{Chapters 14 through 15}.  These chapters contain advanced concepts that are there to allow students to demonstrate mastery of the second part and to perform at the top of their class.  These sections also provide a framework for preparation in competitive mathematics, such as the AMC, Mu Alpha Theta, etc.
\end{enumerate}

\subsection{A Note on Problem Solving}
We believe that the best way to learn mathematics is to solve problems.  Our goal is to have you solve lots of problems in this book, as you will solve many more during your time as a mathematics learner.  We find that the best way to solve problems is to attempt to solve a problem that you don't know how to do by walking through a framework (an outline, per se) of how to do the problem.  That way, you are discovering the solution on your own, and you are bound to remember it better.

This text will also go through much of the required derivations that are essential for success.  We believe that, by deriving all formulas, you will better understand where the math comes from and will have an easier time applying the math.  Physics is an applied math; thus, to perform well in physics, it is important that we cover the information in Algebra and Pre-Calculus so it can be applied.

One of our most important recommendations is to not memorize every formula; it is much more valuable to understand where formulas come from.  We will point out formulas or ideas that are worth memorizing.  Other ideas should be reconsidered for each problem.
\section{Notation in this Text}
We firmly believe that consistent notation will lead to less mistakes in problem-solving; thus, we have created a consistent set of notations that will be used throughout the text.  There are some notations that some students don't like that are used in this text.  It is important that students adjust to equations that aren't "neat".  Below is an incomplete version of this list: \begin{enumerate}
    \item \textbf{All} functions will be written with the independent variables in parentheses.  For example, $y(x)$, $f(y)$, $u(x,y)$ are acceptable notations.  $y$, $f$, and $u$ are not.
    \item For the logarithmic function section, the natural logarithm is denoted as $\log(x)$.  The common logarithm is denoted as $\log_{10}(x)$.
    \item Given a list of exponential functions of the same base, they are written in order of increasing power.  For example, $f(x)=e^{-2x}+e^{3x}-e^{7x}$ is an acceptable notation.  If there is more than one base, write the bases in ascending order as well.
    \item Solutions are written in ascending order.  For example, given two solutions $x_1$ and $x_2$, $x_1<x_2$.
    \item Ensure that parentheses are large enough to encompass the biggest function.  This is easiest done by leaving an open space for the parentheses, writing the argument, then drawing the parentheses.  Otherwise, it can be tough to tell what's inside the parentheses.
    \item For trigonometry, inverse functions are written with a negative exponent: $\sin^{-1}(x)$, $\cos^{-1}(x)$, etc.  are used rather than $\arcsin(x)$ and $\arccos(x)$.
    \item Operators are written with parentheses.  $\sin x$ is not acceptable, while $\cos(x)$ is.
    \item "$:=$" means "equal to by definition".
\end{enumerate}

\section{Acknowledgements}
We would like to thank the MSE Department, especially Mr$.$ Aaron Keevey, Mr$.$ Randal Oddi, Mrs$.$ Beth Pearson, Mrs$.$ Monica Russell, and Mrs$.$ Valerie Newcomer for making Jumpstart a reality.  This program has helped hundreds of students perform well at Suncoast and beyond, making this opportunity like no other.  Thank you to all volunteer helpers that spend their summertime helping the incoming Freshman class to prepare for their Freshman year.  Lastly, we'd like to thank the Media Center Specialists, Mrs$.$ Amy Armbruster and Ms$.$ Catherine Davis, for generously allowing the use of the media center to host Jumpstart.

We would also like to cite and thank various sources from which we found problems for all chapters of this textbook.  These sources include Kuta Software, the American Mathematics Competition (AMC), the American International Mathematics Examination (AIME), various country mathematics competitions (such as Sweden, Morocco, etc.), and the Art of Problem Solving's Alcumus page.

\section{Text Updates}
A list of text updates will be listed here.  Currently, as Version 1.0, there are no updates.  Any modifications to the novel will be listed in here.

\section{Contact Information}
It is important that you reach out to someone at the first realization of difficulty.  If you have specific class-related questions, contact your teacher.  Their syllabus will have the best way to contact them (usually, it's email).

If you believe there to be an error in this text, you may reach out to one of the writers with the error.  Reach out to either Mr.  Cole Ellis \href{mailto:educationelite1@gmail.com}{here} or Mr.  Joshua Kuffour \href{mailto:jkuffour1.jk@gmail.com}{here}.

\section{How We Wrote this Book}
This book was written using the \LaTeX{} document processing system.  We would like to thank the various \LaTeX{} packages we've used along the way in making this document as well as the numerous help sites in helping us define our own document class.  The diagrams were prepared using the Tikz package.

\section{About Us}
We are a group of passionate mathematics enthusiasts that find high value in sharing our knowledge with the world.  Cole Ellis was the lead author of this text and primary editor of the solutions.  Joshua Kuffour, Jonathan Hartman, and Matthew Schrank prepared a draft of sections of the text and thorough notes on other portions.  They also selected and organized many of the problems in text and wrote a draft of many of the solutions.  We sincerely hope that this text will aide you in this tough transition between middle school and high school!

\section{Final Remarks}
We hope that you find this text to be a good resource for you during your time before Suncoast and your time at Suncoast.  We hope that you take advantage of all parts of this book, from beginning to end, to ensure your success at Suncoast and beyond.  Good habits start early; make time each day to review the material and attempt the practice problems provided.  

We had a great time writing this.  Not only does it give us alumni the opportunity to leave a legacy at Suncoast, it gives us a chance to give back to the academic community.  We believe that this community is what pushed us to reach out aspirations, and we hope that this book can be the first stepping stone for you.  Good luck to everyone and we wish the very best for all of you!
\end{document}